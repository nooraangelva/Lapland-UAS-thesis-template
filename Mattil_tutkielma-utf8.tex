% !TEX encoding = UTF-8 Unicode 
\begin{filecontents*}{\jobname.xmpdata}
\Title{Opinnäytetyö}
\Author{Noora Angelva}
\end{filecontents*}

% Yllä on määritelty pdf/a-muotoon vaadittu metadata minimaalisella tavalla
%Tämän LaTeX-pohjan laadintaan ovat osallistuneet Mika Hirvensalo, Teemu Pirttimäki, Petriina Paturi ja Vesa Halava
\documentclass[a4paper,oneside]{article} 
\usepackage{anyfontsize}
% yksipuolinen
\usepackage[finnish]{babel}            				
% suomenkielinen tavutus ja sanasto
\usepackage[T1]{fontenc}               				
% valitaan ääkkösfonttikoodaus
\usepackage[utf8]{inputenc}        						
% skandit utf-8 koodauksella
\usepackage{amsthm}														% theorem- yms. ympäristöt
\usepackage{amsmath}													% AMS-matematiikkatoimintoja
\usepackage{graphicx}           							
% kuvat
%\usepackage[dvips]{graphicx}           			
% ps-kuvat
\usepackage{graphicx,wrapfig,lipsum}

% Seuraavat rivit määrittävät LaTeXin oletusfontin Computer Modern sijasta käytettäväksi New TX-fontin. Poista 
% rivien ensimmäinen % jos tahdot käyttää New TX-fonttia.
%%%%%%%%%%%%%%%%%%%%%%%%%%%%%%%%%%%%%%%%%%%%%%%%
%\usepackage{newtxtext}       
%\usepackage{newtxmath}
%%%%%%%%%%%%%%%%%%%%%%%%%%%%%%%%%%%%%%%%%%%%%%%%
\usepackage[tagpdf]{axessibility} %Saavutettavuuspaketti, kaavoista tulee lukuohjelmalla luettavampia
%HUOM: LaTeXin erillinen accessibility-paketti ei toimi amsthm-paketin kanssa.
%%%%%%%%%%%%%%%%%%%%%%%%%%%%%%%%%%%%%%%%%%%%%%%%
% Seuraavat rivit määrittävät saavutettavuuden kannalta riittävän pdf/a-muodon UTUGradu-järjestelmään.
% Gradua voi olla mukavampi työstää ilman näitä, ne voi ottaa käyttöön vasta 

%%%%%%%%%%%%%%%%%%%%%%%%%%%%%%%%%%%%%%%%%%%%%%%%

\usepackage[a-3b]{pdfx}  											% Lähtökohtana pdf/a-3b
\usepackage[pdfa]{hyperref}										% pdf/a-muotoa varten



%%%%%%%%%%%%%%%%%%%%%%%%%%%%%%%%%%%%%%%%%%%%%%%%
% Testauksen vuoksi pseudolatinaa tuottava
% paketti. Et tarvitse tätä omassa työssäsi.
\usepackage{lipsum}  
%%%%%%%%%%%%%%%%%%%%%%%%%%%%%%%%%%%%%%%%%%%%%%%%
% Fontti määritys
%\renewcommand{\familydefault}{\sfdefault}
%\usepackage[scaled=1]{Arial}
%\usepackage[Arial]{sfmath}
%\everymath={\sf}

\renewcommand{\rmdefault}{phv} % Arial
\renewcommand{\sfdefault}{phv} % Arial
%%%%%%%%%%%%%%%%%%%%%%%%%%%%%%%%%%%%%%%%%%%%%%%%

% A4 mitat ovat 210x297 (mm), ylämarginaali 30 mm, vasen marginaali 30 mm

% Lapin AMK
% 40 mm vasen reuna lapinamk
% 20 mm ylä- ja oikreuna lapinamk
% 210 - 40 - 20 = 150 leveys
% 297 - 20 -20 = 257 korkeus
%%%%%%%%%%%%%%%%%%%%%%%%%%%%%%%%%%%%%%%%%%%%%%%%
\usepackage{geometry}
\geometry{
 a4paper,
 total={150mm,257mm},
 left=40mm,
 top=20mm,
 }
% asettelu!!!!!!!!!!!!
%https://en.wikibooks.org/wiki/LaTeX/Page_Layout

%Suomenkieliset ympäristöt: 

\theoremstyle{plain}
\newtheorem{theorem}{Lause}
\newtheorem{lemma}{Lemma}
\newtheorem{corollary}{Seuraus}
%
\theoremstyle{definition}
\newtheorem{definition}{M\"a\"aritelm\"a}
\newtheorem{example}{Esimerkki}
%
\theoremstyle{remark}
\newtheorem{remark}{Huomautus}

%%%%%%%%%%%%%%%%%%%%%%%%%%%%%%%%%%%%%%%%%%%%%%%%
%Määrittele tässä aika, työn, kirjoittajan ja ohjaajien tiedot

\newcommand{\tekija}{{Noora Angelva}} %tekijän nimi
\newcommand{\Ohjaajat}{{Maisa Mielikäinen ja Tuija x}} %ohjaajat
\newcommand{\Toimeksiantaja}{{Yhtiö X}} %tekijän nimi
\newcommand{\titteli}{{ }} %voi jättää tyhjäksi
\newcommand{\otsikko}{{Headline}}   %Gradun otsikko
\newcommand{\tutkielma}{{Thesis}}
\newcommand{\opinnaytetyo}{{Opinäytetyö}} %Opinnatetyo
\newcommand{\aika}{{2021}}   %vuosi
\newcommand{\tutkintonimike}{{Engineer of Information and Communication Technology}}
\newcommand{\tutkintonimikeFin}{{Tieto- ja viestintätekniikan insinööri}}
\newcommand{\Koulutus}{{Engineer (Bachelor)}}
\newcommand{\KoulutusFin}{{Insinööri (AMK)}}
\newcommand{\koulutus}{{Koulutus}} %Koulutus
\newcommand{\alaotsikko}{{alaotsikko}}
\newcommand{\taustaprojekti}{{mahd. taustaprojekti}}
%%%%%%%%%%%%%%%%%%%%%%%%%%%%%%%%%%%%%%%%%%%%%%%%
%Dokumentin aloitus

\begin{document}
\pagenumbering{arabic} %Saavuttavuuden takija alussa sivunurot roomalaisittain, tutkielman alusta arabialaisittain.
\pagestyle{empty}  %ei sivunumeroa sivun alareunaan

\begin{center}
\includegraphics[width=6cm]{UAS_Logo}
\end{center}

\vspace{8.5cm}
\begin{center}\fontsize{16pt}{1pt}\selectfont
\otsikko \\ 
\vspace{0.7cm}
\fontsize{14pt}{1pt}\selectfont
\alaotsikko
\end{center}

\vspace{5cm}
\begin{center}

\taustaprojekti\\
\vspace{0.7cm}
\tekija \\
\vspace{0.7cm}
\opinnaytetyo \\
\koulutus\\
\tutkintonimike

\end{center}

\begin{center}\fontsize{16pt}{1pt}\selectfont
\aika
\end{center}

% TIIVISTELMÄ (EN) %

\newpage\null
\pagestyle{empty}  %ei sivunumeroa sivun alareunaan

% YLATUNNISTE
\begin{minipage}{0.4\textwidth}
\includegraphics[width=3.7cm]{ylatunnisteLogo}
\end{minipage}
\begin{minipage}{0.6\textwidth}\raggedleft
Abstract of Thesis\\
\end{minipage}

\Koulutus \\
\indent\tutkintonimike \\
\rule{\textwidth}{.1mm}\\

% YLATUNNISTE END
%korjaa taulukon näköseksi
\noindent \textbf{Author} \	\tekija\
Year \	\aika \\
\noindent \textbf{Supervisor(s)}	\ \Ohjaajat \\
\noindent \textbf{Commissioned by}	\ \Toimeksiantaja \\
\noindent \textbf{Subject of thesis} \	\opinnaytetyo \\
\noindent \textbf{Number of pages} \	XX + X \\
\rule{\textwidth}{.1mm}\\

%TIIVISTELMÄ (EN)->

\noindent Kirjoita tähän tiivistelmä. Laita $\backslash${noindent}-komento kappaleen
alkuun, niin \LaTeX\, ei sisennä ensimmäistä riviä.

\vspace{7mm}\noindent Komento $\backslash$vspace taasen jättää sopivan välin kappaleiden väliin - 4mm näyttää aika hyvältä.

\vspace{7mm}\noindent Kirjoita tiivistelmä napakasti ja kaikenlaista toistoa välttäen.

%sivun loppuun ->
\vspace{7mm}\noindent Keywords: tiivistelmäsivu, Pro gradu -tutkielma, \LaTeX-ladontajärjestelmä.


% TIIVISTELMÄ (FIN) %

\newpage\null
\pagestyle{empty}  %ei sivunumeroa sivun alareunaan

% YLATUNNISTE
\begin{minipage}{0.4\textwidth}
\includegraphics[width=3.7cm]{ylatunnisteLogo}
\end{minipage}
\begin{minipage}{0.6\textwidth}\raggedleft
Opinnäytetyön tiivistelmä\\
\end{minipage}\

\indent\KoulutusFin \\
\indent\tutkintonimikeFin \\
\rule{\textwidth}{.1mm}\\
% YLATUNNISTE END

\noindent \textbf{Tekijä} \	\tekija\
\noindent Vuosi \	\aika \\
\noindent \textbf{Ohjaaja(t)}	\ \Ohjaajat \\
\noindent \textbf{Toimeksiantaja}	\ \Toimeksiantaja \\
\noindent \textbf{Työn nimi} \	\opinnaytetyo \\
\noindent \textbf{Sivu- ja liitesivumäärä} \	XX + X \\
\noindent\rule{\textwidth}{.1mm}\\


% TIIVISTELMä ->

\noindent Kirjoita tähän tiivistelmä. Laita $\backslash${noindent}-komento kappaleen
alkuun, niin \LaTeX\, ei sisennä ensimmäistä riviä.

\vspace{7mm}\noindent Komento $\backslash$vspace taasen jättää sopivan välin kappaleiden väliin - 4mm näyttää aika hyvältä.

\vspace{7mm}\noindent Kirjoita tiivistelmä napakasti ja kaikenlaista toistoa välttäen.

%sivun loppuun->
\vspace{7mm}\noindent Asiasanat: tiivistelmäsivu, Pro gradu -tutkielma, \LaTeX-ladontajärjestelmä.


% SISALLYSLUETTELO %
\newpage\null
\tableofcontents
%Aja laTeX-käännös uudelleen saadaksesi ajantasaisen sisällysluettelon
%\cleardoublepage


% TEKSTI OSUUS %

% UUSI SIVU
\newpage\null
\pagestyle{plain} 

ALKUSANAT
% korjaa otsikko!!!!

Tässä luvussa kiitetään.

% UUSI SIVU
\newpage\null
\pagestyle{plain} 

KÄYTETYT MERKIT JA LYHENTEET
% korjaa otsikko!!!!

UTUGradu-järjestelmän ja PDF\LaTeX :n pdf-formaattivaatimusten takia kannattaa käyttää muita kuin eps-muotoisia kuvia. Esimerkiksi pdf-muoto käy mainiosti vektorikuville, kunhan on samaa pdf/a-3b muotoa kuin tämän tiedoston tuottama pdf-tiedosto. Parhaiten toimivat jpg-muotoiset kuvat.


% UUSI SIVU
\newpage\null
\pagestyle{plain} 

\section{JOHDANTO}
\lipsum[1-5]

\section{KÄSITTELYOSA}

\subsection{Saavutettavuus kuvateksteissä}

Kaikkiin tutkielmassa esiintyviin kuviin tulee viitata tekstissä (ks.~\ref{kuvatus1}) Lisäksi kuvatekstin tulee olla kuvaileva, koska saavutettavuuteen liittyvät alt-tekstit eivät oikein toimi käytetyn \LaTeX :n axessibility-paketin kanssa. Jos kuvan perusteella tehdään päätelmä, tulee päätelmä kuvata tekstissä. 
\begin{theorem} Oletetaan että välillä $[a,b]$ on $F'(x)=f(x)$ ja että $f$ on jatkuva.\footnote{Jatkuvuus on oleellinen seikka tässä yhteydessä.} Tällöin
\[
\int_a^bf(x)\,dx=F(b)-F(a).
\]
\end{theorem}

\subsection{Analyyttistä lukuteoriaa}

\begin{figure}[ht]
\begin{center}
\includegraphics[width=.45\textwidth]{siivet.jpg}
\end{center}
\label{kuvatus1}
\caption{Geronon lemniskaatta välillä $[-1,1]$.}
\end{figure}

Seuraavaksi esitellään niin sanottu {\em Riemannin hypoteesi}, jonka mukaan ns. $\zeta$-funktion epätriviaalit nollakohdat sijaitsevat suoralla $\operatorname{Re}z=\frac12$.
\begin{definition} Riemannin $\zeta$-funktio määritellään sarjaesityksen \cite{Riemann}
\[
\zeta(z)=\sum_{n=1}^{\infty}\frac{1}{n^z}
\]
avulla, kun $\operatorname{Re}z>1$.
\end{definition}

\subsubsection{Matriisilaskentaa}

Nyt tarkastellaan matriisin
\[
A=\left(\begin{array}{rrr} -1 & -2 & -5\\ 3& 4& 5\\ -3 & 2 & 1 \end{array}\right)
\]
ominaisarvoja.



\section{POHDINTA}
\lipsum[1-3]


%Kirjallisuusluettelon määrittelyssä {99} on levein kirjallisuusviitteen numero. Oikaise tarvittaessa.
 
\begin{thebibliography}{99}

\bibitem{NewtLeib} Newton ja Leibniz

\bibitem{Riemann} Riemann

\end{thebibliography}

\section{LIITTEET}

Tässä luvussa esitetään niin kutsuttu Newtonin-Leibnizin kaava \cite{NewtLeib}.

\end{document}
